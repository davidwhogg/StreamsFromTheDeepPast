\documentclass[12pt,letterpaper]{article}
\usepackage{amsmath,amssymb,mathrsfs}

\newcommand{\mvector}[1]{\boldsymbol{#1}}
  \newcommand{\vr}{\mvector{r}}
  \newcommand{\vv}{\mvector{v}}
  \newcommand{\vx}{\mvector{x}}
  \newcommand{\vz}{\mvector{z}}
  \newcommand{\observable}{\mvector{w}}
  \newcommand{\zero}{\mvector{0}}
\newcommand{\vhat}[1]{\mvector{\hat{#1}}}
  \newcommand{\rhat}{\vhat{r}}
  \newcommand{\thetahat}{\vhat{\theta}}
  \newcommand{\phihat}{\vhat{\phi}}
  \newcommand{\zhat}{\vhat{z}}
\newcommand{\dd}{\mathrm{d}}
\newcommand{\Dec}{\mathrm{Dec}}
\newcommand{\abs}[1]{|{#1}|}
\newcommand{\cross}{\times}

\begin{document}

\paragraph{Definitions:}
We will take the 6-dimensional phase-space vector to be made up of two
3-dimensional position and velocity vectors $[\vx,\vv]$.  In the
position space there are three orthogonal dimensionless unit vectors
$\rhat$, $\thetahat$, and $\phihat$, pointing in the radial,
polar-angle, and azimuthal-angle directions in spherical coordinates.
We will take the observables to be the set of six quantities
\begin{eqnarray}\displaystyle
\observable &\equiv& [D,\theta,\phi,v_r,\mu_\theta,\mu_\phi]
\quad ,
\end{eqnarray}
where $D$ is the distance (dimensions of length), $\theta, \phi$ are
the angular coordinates on the sphere (dimensions of angle or
dimensionless), $v_r$ is the radial velocity (dimensions of length per
time), and $\mu_\theta, \mu_\phi$ are the angular velocities in the
direction of increasing $\theta$ and $\phi$ (units of angle per time
or just inverse time).

Trivially, the distance and radial velocity are given by unit-vector
dot products
\begin{eqnarray}\displaystyle
D &\equiv& \vx\cdot\rhat
\nonumber\\
v_r &\equiv& \vv\cdot\rhat
\quad .\label{eq:radials}
\end{eqnarray}
The $\phi$-direction angular velocity has the pesky $\sin\theta$ (or
$\cos\theta$, depending on definition of $\theta$ on the equator)
factor divided out, so that the angular velocities are just given by
\begin{eqnarray}\displaystyle
\mu_{\theta} &\equiv& \frac{1}{D}\,\vv\cdot\thetahat
\nonumber\\
\mu_{\phi} &\equiv& \frac{1}{D}\,\vv\cdot\phihat
\quad .\label{eq:pms}
\end{eqnarray}

If a star is observed to lie at position $\vx$, we will define for
that star the unit vectors $\rhat$, $\thetahat$, and $\phihat$ such
that
\begin{eqnarray}\displaystyle
0 &=& \vx\cdot\thetahat
\nonumber\\
0 &=& \vx\cdot\phihat
\quad .\label{eq:alignment}
\end{eqnarray}
This will zero out some of the pesky terms in our derivatives.

\paragraph{Derivatives:}
We have aligned the unit vectors with the position according to
equation~(\ref{eq:alignment}), but now we are going to take
derivatives of the phase-space coordinates with respect to the
observables, with the unit vectors $\rhat$, $\thetahat$, and $\phihat$
\emph{held fixed}.  That is, there will be no derivatives of the unit
vectors with respect to observables or phase-space position.

The system is ``sparse'' in the sense that most of the observables
touch only very well-defined linear combinations of the phase-space
components.  For example, $\mu_{\theta}$ appears only in the
projection $\vv\cdot\thetahat$.  The exception is the distance $D$,
which is involved in all four of the angular quantities $[\theta,
  \phi, \mu_{\theta}, \mu_{\phi}]$.  So the $D$ derivative of the
phase-space coordinates is the most complicated.  It would be
exceedingly complicated if it were not for the fact that we have
aligned the unit vectors according to equation~(\ref{eq:alignment}),
so that some of the cross terms (those including $\vx\cdot\thetahat$
and $\vx\cdot\phihat$) will drop to zero.

When I use my intuitive derivative-taking magic on this system I get:
\begin{eqnarray}\displaystyle
\frac{\dd}{\dd D}[\vx, \vv] &=& [\rhat, \mu_\theta\,\thetahat+\mu_\phi\,\phihat]
\nonumber\\
\frac{\dd}{\dd \theta}[\vx, \vv] &=& [D\,\thetahat, \zero]
\nonumber\\
\frac{1}{\sin\theta}\,\frac{\dd}{\dd \phi}[\vx, \vv] &=& [D\,\phihat, \zero]
\nonumber\\
\frac{\dd}{\dd v_r}[\vx, \vv] &=& [\zero, \rhat]
\nonumber\\
\frac{\dd}{\dd \mu_\theta}[\vx, \vv] &=& [\zero, D\,\thetahat]
\nonumber\\
\frac{\dd}{\dd \mu_\phi}[\vx, \vv] &=& [\zero, D\,\phihat]
\quad ,\label{eq:derivatives}
\end{eqnarray}
where $\zero$ is the 3-dimensional zero vector, and the $1/\sin\theta$
factor makes the $\dd /\dd\phi$ derivative isotropic on the sphere.
If you want these expressions in terms of phase-space coordinates
instead of observables, you can convert with the
expressions~(\ref{eq:radials}) and (\ref{eq:pms}).

These 6 rows (\ref{eq:derivatives}) of 6-vectors make up the full
$6\times 6$ derivative matrix $\dd [\vx, \vv] / \dd\observable$; the
opposite derivative matrix $\dd\observable / \dd [\vx,\vv]$ is
obtained (except maybe at $D=0$ or maybe at the poles of the sphere)
by taking the matrix inverse.

\end{document}
