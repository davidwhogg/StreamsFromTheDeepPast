%\documentclass[landscape,a0b,final,a4resizeable]{a0poster}
%\documentclass[landscape,a0b,final]{a0poster}
%\documentclass[portrait,a0b,final,a4resizeable]{a0poster}
\documentclass[portrait,a0b,final]{a0poster}
%%% Option "a4resizeable" makes it possible ot resize the
%   poster by the command: psresize -pa4 poster.ps poster-a4.ps
%   For final printing, please remove option "a4resizeable" !!
\usepackage{amsmath}
\usepackage{shadow}
\usepackage{epsfig}
\usepackage{pstricks,pst-grad}
\usepackage{multicol}
\usepackage{float}

%%%%%%%%%%%%%%%%%%%%%%%%%%%%%%%%%%%%%%%%%%%
% Definition of some variables and colors
%\renewcommand{\rho}{\varrho}
%\renewcommand{\phi}{\varphi}
\newcommand{\tvector}[1]{\boldsymbol{\vec{#1}}}
\newcommand{\vx}{\tvector{x}}
\newcommand{\vv}{\tvector{v}}
\newcommand{\setofxv}{\{\vx_i,\vv_i\}}
\newcommand{\mvector}[1]{\boldsymbol{{#1}}}
\newcommand{\mvtheta}{\mvector{\theta}}
\newcommand{\mvomega}{\alpha,\ln A}
\newcommand{\dd}{\mathrm{d}}
\setlength{\columnsep}{1cm}
\setlength{\columnseprule}{1.5mm}
\setlength{\parindent}{0.0cm}



%%%%%%%%%%%%%%%%%%%%%%%%%%%%%%%%%%%%%%%%%%%%%%%%%%%%
%%%                Poster                        %%%
%%%%%%%%%%%%%%%%%%%%%%%%%%%%%%%%%%%%%%%%%%%%%%%%%%%%

\newenvironment{poster}{
  \begin{center}
  \begin{minipage}[c]{0.98\textwidth}
}{
  \end{minipage} 
  \end{center}
}



%%%%%%%%%%%%%%%%%%%%%%%%%%%%%%%%%%%%%%%%%%%%%%%%%%%%
%%%                pcolumn                       %%%
%%%%%%%%%%%%%%%%%%%%%%%%%%%%%%%%%%%%%%%%%%%%%%%%%%%%

\newenvironment{pcolumn}[1]{
  \begin{minipage}{#1\textwidth}
  \begin{center}
}{
  \end{center}
  \end{minipage}
}



%%%%%%%%%%%%%%%%%%%%%%%%%%%%%%%%%%%%%%%%%%%%%%%%%%%%
%%%                pbox                          %%%
%%%%%%%%%%%%%%%%%%%%%%%%%%%%%%%%%%%%%%%%%%%%%%%%%%%%

\newrgbcolor{lcolor}{0. 0. 0.80}
\newrgbcolor{gcolor1}{1. 1. 1.}
\newrgbcolor{gcolor2}{.80 .80 1.}

\newcommand{\pbox}[4]{
\psshadowbox[#3]{
\begin{minipage}[t][#2][t]{#1}
#4
\end{minipage}
}}



%%%%%%%%%%%%%%%%%%%%%%%%%%%%%%%%%%%%%%%%%%%%%%%%%%%%
%%%                myfig                         %%%
%%%%%%%%%%%%%%%%%%%%%%%%%%%%%%%%%%%%%%%%%%%%%%%%%%%%
% \myfig - replacement for \figure
% necessary, since in multicol-environment 
% \figure won't work

\newcommand{\myfig}[3][0]{
%\begin{center}
%  \vspace{1.5cm}
  \includegraphics[width=#3\hsize,angle=#1]{#2}
%  \nobreak\medskip
%\end{center}}
}


%%%%%%%%%%%%%%%%%%%%%%%%%%%%%%%%%%%%%%%%%%%%%%%%%%%%
%%%                mycaption                     %%%
%%%%%%%%%%%%%%%%%%%%%%%%%%%%%%%%%%%%%%%%%%%%%%%%%%%%
% \mycaption - replacement for \caption
% necessary, since in multicol-environment \figure and
% therefore \caption won't work

%\newcounter{figure}
\setcounter{figure}{1}
\newcommand{\mycaption}[1]{
  \vspace{0.5cm}
  \begin{quote}
    {{\sc Figure} \arabic{figure}: #1}
  \end{quote}
  \vspace{1cm}
  \stepcounter{figure}
}


\newrgbcolor{lightblue}{0. 0. 0.80}
\newrgbcolor{white}{1. 1. 1.}
\newrgbcolor{whiteblue}{.8 .8 1.}

%%%%%%%%%%%%%%%%%%%%%%%%%%%%%%%%%%%%%%%%%%%%%%%%%%%%
%%%               Background                     %%%
%%%%%%%%%%%%%%%%%%%%%%%%%%%%%%%%%%%%%%%%%%%%%%%%%%%%

\newcommand{\background}[3]{
  \psframe[fillstyle=gradient,gradbegin=white,gradend=white,gradmidpoint=1.](-5.,5.)(1.2\textwidth,-1.2\textheight)
}

%%%%%%%%%%%%%%%%%%%%%%%%%%%%%%%%%%%%%%%%%%%%%%%%%%%%%%%%%%%%%%%%%%%%%%
%%% Begin of Document
%%%%%%%%%%%%%%%%%%%%%%%%%%%%%%%%%%%%%%%%%%%%%%%%%%%%%%%%%%%%%%%%%%%%%%

\begin{document}

\background{1. 1. 1.}{1.}{0.5}

\vspace*{2cm}



%\begin{poster}

%%%%%%%%%%%%%%%%%%%%%
%%% Header
%%%%%%%%%%%%%%%%%%%%%
\begin{center}
\begin{pcolumn}{0.98}

\pbox{0.95\textwidth}{}{linewidth=2mm,framearc=0.3,linecolor=lightblue,fillstyle=gradient,gradangle=0,gradbegin=white,gradend=whiteblue,gradmidpoint=1.0,framesep=1em}{

%%% Titel
\begin{minipage}[c][9cm][c]{0.78\textwidth}
  \begin{center}
    {\sc \Huge The velocity distribution of nearby stars}\\[10mm]
    {\sc \Huge from \emph{Hipparcos} data}\\[10mm]
    {\Large Jo Bovy$^1$, David W.~Hogg$^{1,2}$, \& Sam Roweis$^{3,4}$}\\[2mm]
    {\large \phantom{q}$^1$ New York University,
      \phantom{q}$^2$ Max-Planck-Institut f\"{u}r Astronomie,
      \phantom{q}$^3$ University of Toronto,
      \phantom{q}$^4$ Google Inc.}
  \end{center}
\end{minipage}
%%% Logo
\begin{minipage}[c][9cm][c]{0.15\textwidth}
  \begin{center}
    \includegraphics[width=18cm,angle=0]{nyulogoblack.eps}
  \end{center}
\end{minipage}

}
\end{pcolumn}
\end{center}


\vspace*{2cm}



%%% Begin of Multicols-Enviroment
\begin{multicols}{2}
%%% Introduction
%\vspace{2cm}\begin{center}\pbox{0.4\columnwidth}{}{linewidth=2mm,framearc=0.1,linecolor=lightblue,fillstyle=gradient,gradangle=0,gradbegin=white,gradend=whiteblue,gradmidpoint=1.0,framesep=1em}{\begin{center}{\Huge Introduction}\end{center}}\end{center}\vspace{1.25cm}


{\Large $\bullet$ Using a deconvolution approach we fit a
three-dimensional model for the underlying velocity distribution to
the observed tangential velocities and their uncertainties of a sample
of 11,865 main sequence stars from \emph{Hipparcos}. This model
consists of a mixture of $K$ Gaussian distributions, with a
regularization prior $w$ on the variances of the Gaussians (a weak
Inverse-Wishart prior). For different combinations of these two
parameters we find the best fit velocity distribution:}
%%% Figures:
\begin{center}
  % first argument: eps-file
  % second argument: stretching-factor relative to Column-width (<1)
  % optional argument: rotation angle (0-360), default=0
  \myfig{veldensXY.ps}{1}\\[15pt]
%  \mycaption{}
\end{center}



{\Large $\bullet$ For each ($K$,$w$) pair we predict the radial
velocities of stars in the \emph{Geneva-Copenhagen} radial velocity
survey (GCS) using our best-fit model and compare these to the
observed radial velocities. An example for a few stars for $K$=10,
$w$=4 km$^2$ s$^{-2}$:}
\begin{center}
  % first argument: eps-file
  % second argument: stretching-factor relative to Column-width (<1)
  % optional argument: rotation angle (0-360), default=0
  \myfig{predict_gcs_rand.ps}{.9}%\\[2cm]
%  \mycaption{}
\end{center}
Note: the black curve is the prediction for the radial velocity
conditioned on the measured tangential velocity, while the gray curve
is not. The black vertical line is the observed velocity and the gray
vertical lines are 95\,percent confidence limits for the conditioned
prediction.

{\Large $\bullet$ The preferred underlying model is that which
predicts the radial velocities best (lower right panel below). A
number of \emph{internal} validation tests are presented here as well,
most of which \emph{overestimate} the amount of structure as compared
with the GCS validation.}
\begin{center}
  % first argument: eps-file
  % second argument: stretching-factor relative to Column-width (<1)
  % optional argument: rotation angle (0-360), default=0
  \myfig{crossval.ps}{0.31}
  \myfig{aic.ps}{0.31}
  \myfig{mdl.ps}{0.31}\\[.7cm]
  \myfig{mml.ps}{0.31}
  \myfig{bayes.ps}{0.31}
  \myfig{gcs_cross.ps}{0.31}\\[1cm]
\end{center}
Note: Shown are (1) leave-one-percent-out cross-validation; (2) entropy
based Akaike's information criterion (AIC); Minimum coding inference
based (3) minimum description length (MDL) and (4) minimum message
length (MML); (5) The Bayesian evidence; (6) total probability of the
predicted GCS radial velocities.\\[.5cm]



{\Large $\bullet$ The resulting distribution contains all of the classical moving groups, but nothing beyond those:}
\begin{center}
  % first argument: eps-file
  % second argument: stretching-factor relative to Column-width (<1)
  % optional argument: rotation angle (0-360), default=0
  \myfig{annotated_veldist.ps}{1}%\\[1cm]
%  \mycaption{}
\end{center}
Note: Contours and grayscalesn are linear; 50\,percent of the
distribution is contained within the innermost dark contour.\\[.2cm]

{\large {\bf References:\\} 
Jo Bovy, David W.~Hogg, \& Sam T.~Roweis, \emph{The velocity distribution of nearby stars from \emph{Hipparcos} data I. The significance of the moving groups}, ApJ, in press, arXiv:0905.2980 [astro-ph]\\
Jo Bovy, David W.~Hogg, \& Sam T.~Roweis, \emph{Extreme deconvolution: inferring complete distribution functions from noisy, heterogeneous, and incomplete observations}, arXiv:0905.2979 [stat.ME]\\[1.5cm]
}

{\large {\bf Acknowledgments:} This research was partially supported by NASA (ADP grant NNX08AJ48G)}\\[1cm]

\end{multicols}

%\end{poster}

\end{document}

