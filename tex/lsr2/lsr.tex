\documentclass[12pt,preprint]{aastex}
\usepackage{amssymb,amsmath}
%\usepackage{color,hyperref}
\newcounter{address}
\setcounter{address}{1}
% hypertex insanity
%\definecolor{linkcolor}{rgb}{0,0,0.25}
%\hypersetup{
%  colorlinks=true,        % false: boxed links; true: colored links
%  linkcolor=linkcolor,    % color of internal links
%  citecolor=linkcolor,    % color of links to bibliography
%  filecolor=linkcolor,    % color of file links
%  urlcolor=linkcolor      % color of external links
%}
\newcommand{\eqnnumber}{equation}
\newcommand{\eg}{e.g.}
\newcommand{\ie}{i.e.}
\newcommand{\Ie}{I.e.}
\newcommand{\etal}{et~al.}
\newcommand{\Hipparcos}{\textit{Hipparcos}}
\newcommand{\Gaia}{\textit{Gaia}}
\newcommand{\lsr}{LSR}
\newcommand{\sgra}{Sgr A$^*$}
\renewcommand{\vec}[1]{\mathbf{#1}} % boldface for vectors
\newcommand{\aag}{\ensuremath{\vec{a}}}

\begin{document}

\title{On the Local Standard of Rest}
\author{
Jo~Bovy\altaffilmark{\ref{NYU},\ref{email}}, 
David~W.~Hogg\altaffilmark{\ref{NYU},\ref{MPIA}}} 
\altaffiltext{\theaddress}{\label{NYU}\stepcounter{address}
Center for Cosmology and Particle Physics, Department of Physics, New York 
University, 4 Washington Place, New York, NY 10003}
\altaffiltext{\theaddress}{\stepcounter{address}\label{email}
To whom correspondence should be addressed: \texttt{jo.bovy@nyu.edu}}
\altaffiltext{\theaddress}{\label{MPIA}\refstepcounter{address}
  Max-Planck-Institut f\"ur Astronomie,
  K\"onigstuhl 17, D-69117 Heidelberg, Germany}

\begin{abstract}
We critically examine measurements of the Local Standard of Rest
(\lsr) and of the Solar motion with respect to the \lsr\ and show that
both are fraught by systematical uncertainties which dominate the
uncertainty budget but are rarely accounted for. We argue that in most
cases of interest the need for these two quantities can be bypassed by
considering the motion of the Sun with respect to the Galactic center,
which can be (BOVY: has been?) established at high precision and with
minimal systematical uncertainty from the apparent motion of \sgra. In
terms of dynamics, we find it more usefull to think of the local
gravitational acceleration instead of the velocity of a hypothetical,
presumably non-existent, circular orbit.
\end{abstract}
\keywords{
Galaxy: fundamental parameters ---
Galaxy: kinematics and dynamics ---
Galaxy: structure ---
methods: statistical ---
Solar neighborhood ---
stars: kinematics
}

\section{Introduction}

The Local Standard of Rest is used a lot
\citep{1998MNRAS.298..387D}. BOVY: CITE some papers that use this,
\eg, the RAVE escape speed paper.


\section{The Solar velocity with respect to the \lsr}

In this section we follow a number of \emph{ad hoc} procedures to deal
with the presence of the moving groups in the velocity
distribution. These ad hoc choices show that the uncertainty in the
Solar velocity with respect to the \lsr\ is dominated by these
systematic modeling uncertainties, and not by statistical errors.



\section{The velocity of the \lsr}

Certainly Reid cannot be trusted. How have other people established
the velocity of the \lsr. What is the underlying assumption of the
Oort's constants method?



\section{Bypassing the \lsr}

By directly measuring the Solar motion with respect to \sgra, the only
remaining uncertainty becomes the distance to \sgra. This is
\emph{much better} and \emph{much simpler} than the two-step process
Sun$\rightarrow$\lsr\ and \lsr$\rightarrow$\sgra.

\section{The local gravitational acceleration}

What we are really measuring is the local gravitational acceleration
\aag\ and not the velocity of a, hypothetical and presumably
non-existent, circular orbit with this acceleration. Thinking about
the local gravitational acceleration instead of about the circular
velocity makes it clear that we are making assumptions about the
potential and the tracers we use to measure it. The local
gravitational acceleration is also something that is always
well-defined in a general gravitational potential, while circular
orbits only exist for a very limited number of potentials, none of
which is likely to describe our Galaxy.

\section{Conclusion}

\bibliographystyle{apj}
\bibliography{../rv/EM,../rv/bovy_streams,../rv/MML}


\end{document}