\documentclass[12pt]{article}
\usepackage{amssymb,amsmath}
\newcommand{\dd}{\mathrm{d}}
\newcommand{\tvector}[1]{\boldsymbol{\vec{#1}}}
\newcommand{\unitz}{\tvector{\hat{z}}}
\newcommand{\vx}{\tvector{x}}
\newcommand{\vv}{\tvector{v}}
\newcommand{\vI}{\tvector{I}}
\newcommand{\vphi}{\tvector{\phi}}
\newcommand{\vomega}{\tvector{\omega}}
\newcommand{\LL}{\tvector{L}}
\newcommand{\rperi}{\ensuremath{r_{\mbox{{\footnotesize peri}}}}}
\newcommand{\rap}{\ensuremath{r_{\mbox{{\footnotesize ap}}}}}
\newcommand{\matrixleft}{\left[}
\newcommand{\matrixright}{\right]}
\newcommand{\freqrfreqphiprop}{I}
\newcommand{\eqnname}{equation}
\newcommand{\T}{^{\scriptscriptstyle \top}}   % transpose
\begin{document}

Remarkably, when a satellite of the Galaxy is tidally stripped---that
is, when stars leave orbits that circumnavigate the satellite and
enter onto orbits that circumnavigate only the larger Galaxy---the
tidal debris generically forms into long, thin structures in
configuration space, hewing close to the original satellite orbit.
This document investigates the theoretical properties of these tidal
streams by asking the following: Does a cold stellar stream created by
tidal stripping of a cold satellite ``trace out'' a test-particle
orbit?  Clearly it does not, exactly, but here I seek to make the
question extremely precise, and answer it.

In what follows, I will assume that the satellite and tidal debris are
of vanishing mass compared to the parent Galaxy, and that the parent
Galaxy potential is time-independent and integrable.  That is, I will
assume that the phase-space position $[\vx,\vv]$ of a particle
orbiting in the Galaxy potential can be written as (canonically
transformed to) new coordinates $[\vI,\vphi]$, where $\vI$ is the
position in the three-dimensional space of conserved
(time-independent) actions and $\vphi$ is the position in the
three-dimensional space of conjugate angles, which cycle linearly with
time at angular velocity $\vomega$.  By construction, the angular
velocity is independent of time (or position in angle space), but in
general it is a function of position in action space.

There are two senses in which one could mean to say that a stream does
\emph{not} trace out a test-particle orbit.  The first (and to my
mind, trivial) sense is that because the different stars in the stream
have different \emph{actions}.  There is, at least at late times, an
action gradient along the stream; this is what causes the stream to
elongate with time.  The existence of this action gradient means that
the stream does not trace out or enclose a single test-particle orbit:
The different stars in the stream are in fact on minutely different
orbits.

The second sense in which a stream might not trace out an orbit is
\emph{in addition} to this; it is the sense in which the
\emph{direction of elongation} of the stream need not be in the
\emph{orbital direction}.  Consider two test particles, $i$ and $j$,
with very slightly different phase-space positions.  For definiteness,
let's define a mean phase-space position
\begin{equation}
[\vI,\vphi] \equiv \frac{1}{2}
 \,\left([\vI_j,\vphi_j] + [\vI_i,\vphi_i]\right) \quad ,
\end{equation}
and a displacement from the mean
\begin{equation}
[\delta\vI,\delta\vphi] \equiv \frac{1}{2}
 \,\left([\vI_j,\vphi_j] - [\vI_i,\vphi_i]\right) \quad ,
\end{equation}
and the equivalents in the $[\vx,\vv]$ coordinates.

Begin (at time $t=0$) with infinitesimal displacements
$[\delta\vI,\delta\vphi(0)]$.  The angle-space part of the phase-space
positions of the particles will diverge; the displacement
$[\delta\vI,\delta\vphi(t)]$ will grow with time $t$.  Will these
particles diverge along the direction of motion; will $\delta\vphi(t)$
grow parallel to $\vomega$ or will $\delta\vx(t)$ grow parallel to
$\vv(t)$?  Again, not \emph{exactly}, since they are on slightly
different orbits.  But we can take the limits
$[\delta\vI,\delta\vphi(0)]\rightarrow [0,0]$ and $t\rightarrow
\infty$ such that the magnitude of the displacement $\delta\vphi(t)$
is kept small but finite.  In this limit, will the line segment
joining the two particles lie along an orbit?

It is particularly instructive to think about this problem in
action--angle space.  In these coordinates, $\delta\vI$ is not a
function of time at all, because the actions are conserved (by
construction).  Divergence of the trajectories is confined to the
angle subspace.  Furthermore, because all particles cycle their angles
linearly, the particles will diverge linearly in the angle subspace;
that is
\begin{equation}
\delta\vphi(t) = \delta\vphi(0) + \delta\vomega\,t \quad ,
\end{equation}
where $\delta\omega$ is a constant angular frequency that can be
computed from the action displacement
\begin{equation}
\delta\vomega = \frac{\dd\vomega}{\dd\vI}\cdot\delta\vI \quad ,
\end{equation}
where we are thinking of the vectors like column vectors, the
derivative is the full $3\times 3$ matrix of derivatives, we have used
the fact that $\vomega$ depends only on the actions $\vI$, and we have
implicitly taken the limit $\delta\vI\rightarrow 0$.  In these
coordinates, the question of whether the displacement grows in the
direction of the stream can be written as follows: Is $\delta\vomega$
parallel to $\vomega$ for generic action displacements $\delta\vI$?

Obviously, the general answer is ``no'' in that we \emph{can} find
action displacements $\delta\vI$ for which the angular velocities
$\vomega$ and $\delta\vomega$ will not be parallel, provided that the
matrix of derivatives has high rank.  However, the question is not a
mathematical one; it is a physical one: One sense in which the answer
might be ``yes'' is if the matrix of derivatives is of low rank;
another is if the matrix of derivatives has has one eigenvalue that is
substantially larger than the others, with an eigenvector parallel to
$\vomega$ (or something like that, no?).  In these cases, although not
\emph{all} pairs of test particles will diverge along the orbit
direction, given a cloud or ball of test particles in phase-space, the
direction that would elongate most quickly would be parallel to the
direction of motion.

Now, is \emph{this} true in general, or at least of the Galaxy?  It
certainly seems to be true in certain limits, for example when
considering nearly circular orbits in a spherical potential (although
in a spherical potential, I am not sure that the matrix of derivatives
is of high rank).  

\paragraph{The relation between $\omega_r$ and
$\omega_\phi$ in a spherical potential} 
We write the energy for a spherical potential $u(r)$ as
\begin{equation}
\epsilon= u(r) + \frac{|\LL|^2}{2\,r^2} + \frac{1}{2}v_r^2\, .
\end{equation}
Choosing coordinates such that $\LL = L_z\,\unitz$, we can take the actions to be
\begin{align}
J_\phi &= L_z\, ,\\
J_\theta &=0\, ,\\
J_r &= \frac{1}{\pi} \int_{\rperi}^{\rap} \dd r\, \frac{1}{r} \sqrt{2[\epsilon-u(r)]\, r^2 - L^2}\, ,
\end{align}
where we have defined $L^2 \equiv |\LL|^2$. In order to find the
orbital frequencies $\omega_i = \frac{\dd H}{\dd J_i}$, where $H$ is
the Hamiltonian, we can consider the following
\begin{align}
\frac{\partial J_r}{\partial H} &= \frac{1}{\pi} \int_{\rperi}^{\rap} \dd r\, \frac{r}{\sqrt{2[\epsilon-u(r)]\,r^2-L^2}}\, ,\\
\frac{\partial J_r}{\partial L_z} &= -\frac{1}{\pi} \int_{\rperi}^{\rap} \dd \,\frac{1}{r}\,\frac{L_z}{\sqrt{2[\epsilon-u(r)]\,r^2-L^2}}\, ,\\
\frac{\partial J_\phi}{\partial H} &= 0\, ,\\
\frac{\partial J_\phi}{\partial L_z} &= 1\, .
\end{align}
Inverting the matrix $\frac{\partial(J_r,J_\phi)}{\partial(H,L_z)}$
gives us
\begin{equation}
\frac{\partial (J_r, J_\phi)}{\partial (H,L_z)} = \matrixleft \begin{array}{cc} \frac{2\,\pi}{T_r} & \frac{2\,\pi}{T_r}\freqrfreqphiprop  \\ 0 & \frac{2\,\pi}{T_r}\end{array}\matrixright\, ,
\end{equation}
where we have defined the radial period
\begin{equation}
T_r \equiv 2\int_{\rperi}^{\rap} \dd r\, \frac{r}{\sqrt{2[\epsilon-u(r)]\,r^2-L^2}}\,
\end{equation}
and the function $\freqrfreqphiprop$
\begin{equation}
\freqrfreqphiprop= \frac{1}{\pi} \int_{\rperi}^{\rap} \dd r\, \frac{1}{r} \, \frac{L_z}{\sqrt{2[\epsilon-u(r)]\,r^2-L^2}}\, .
\end{equation}
Therefore, we see that the following relation holds between the radial
frequency $\omega_r$ and the azimuthal frequency $\omega_\phi$
\begin{equation}
\omega_\phi = I\, \omega_r\, .
\end{equation}


\paragraph{Does a cold stellar stream trace out a test-particle orbit?}
In the simplest case the factor I does not depend on the actions. This
is, for example, the case for a Kepler potential (for which $I=1$), or
for the harmonic oscillator potential (for which $I=1/2$). In this
case a cold stellar stream will trace out an orbit, since the matrix
of derivatives $\frac{\dd\vomega}{\dd\vI}$ is then of the form
\begin{equation}\label{eq:constmatrix}
\matrixleft \begin{array}{cc} A & B\\ A\,I& B\,I\end{array}\matrixright\, ,
\end{equation}
because
\begin{equation}
\frac{\partial \omega_\phi}{\partial J_i} = I\,\frac{\partial
  \omega_r}{\partial J_i} \qquad \qquad i=r,\phi\, .
\end{equation}
The matrix in \eqnname~(\ref{eq:constmatrix}) has eigenvalues 0,
$A+B\,I$, with associated eigenvalues $\matrixleft \begin{array}{cc} B
& -A \end{array}\matrixright\T$ and $\matrixleft \begin{array}{cc} 1
&I\end{array}\matrixright\T$. Therefore, the dominant eigenvector is
in the direction of $\vomega$, while the eigenvalue associated with
the other eigenvalue is zero. Therefore, the stream will trace out an
orbit in all cases.

Now, if the proportionality factor $I$ between $\omega_\phi$ and
$\omega_r$ is not constant, then when
\begin{equation}
\delta \equiv \frac{\partial \ln I}{\partial \ln \omega_r} \ll 0 \qquad \qquad \epsilon \equiv \frac{\partial \ln I}{\partial \ln \omega_\phi} \ll 0\, ,
\end{equation}
then the tidal stream will approximately trace out an orbit. In this
case the eigenvalues are $ABI(\epsilon-\delta)/(A+BI)$ and $(A+BI) +\
\frac{BI}{A+BI}(A\delta+BI\epsilon)$, with associated eigenvectors
$\matrixleft \begin{array}{cc} -\frac{B}{A} +
\frac{B^2I}{2A(A+BI)}(\delta-\epsilon) & 1 \end{array}\matrixright\T$
and $\matrixleft \begin{array}{cc} \frac{1}{I}
+\frac{1}{A+BI}(B\epsilon-A\delta) &1\end{array}\matrixright\T$ (BOVY:
check this).

\paragraph{The case of the logarithmic potential}
The potential $u(r)$ for an isothermal sphere is given by
\begin{equation}
u(r) = v_c(r_0)^2 \ln (\frac{r}{r_0})\, .
\end{equation}
The factor $I$ is then given by
\begin{equation}
I = 
\end{equation}

 \end{document}
