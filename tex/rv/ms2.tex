\documentclass[12pt,preprint]{aastex}

\usepackage{amsmath}

\newcounter{address}
%\newcommand{\latin}[1]{\textit{#1}}
%\newcommand{\cf}{\latin{cf.}}
\newcommand{\eqnnumber}{Equation}
\newcommand{\eg}{e.g.}
\newcommand{\ie}{i.e.}
\newcommand{\Ie}{I.e.}
%\newcommand{\vs}{\latin{vs.}}
\newcommand{\etal}{et~al.}
\newcommand{\Hipparcos}{\textit{Hipparcos}}
\newcommand{\Gaia}{\textit{Gaia}}
\newcommand{\Tycho}{Tycho}
\newcommand{\gcs}{\textit{Geneva-Copenhagen}}
\newcommand{\normal}{{\cal N}}
\newcommand{\wishart}{{\cal W}}
\newcommand{\dirichlet}{{\cal D}}
\renewcommand{\vec}[1]{\mathbf{#1}} % boldface for vectors
\newcommand{\inv}{^{-1}}
\newcommand{\bb}{\vec{b}}
\newcommand{\cc}{\vec{c}}
\newcommand{\ee}{\vec{\hat{e}}}
\newcommand{\mm}{\vec{m}}
\newcommand{\vv}{\vec{v}}
\newcommand{\xx}{\vec{x}}
\newcommand{\zz}{\vec{z}}
\newcommand{\ww}{\vec{w}}
\newcommand{\bij}{\bb_{ij}}
\newcommand{\bbij}{\bij}
%\newcommand{\cci}{\cc_i}
%\newcommand{\eex}{\vec{\hat{x}}}
%\newcommand{\eey}{\vec{\hat{y}}}
%\newcommand{\eez}{\vec{\hat{z}}}
%\newcommand{\eer}{\vec{\hat{r}}_i}
%\newcommand{\eel}{\vec{\hat{l}}_i}
%\newcommand{\eeb}{\vec{\hat{b}}_i}
\newcommand{\mmj}{\mm_j}
\newcommand{\mmk}{\mm_k}
\newcommand{\vvi}{\vv_i}
\newcommand{\vvj}{\vv_j}
%\newcommand{\vvdisk}{\vv_\mathrm{disk}}
%\newcommand{\Kdisk}{K_\mathrm{disk}}
%\newcommand{\vvhalo}{\vv_\mathrm{halo}}
%\newcommand{\vvlsr}{\vv_\mathrm{LSR}}
%\newcommand{\vvsun}{\vv_\odot}
\newcommand{\wwi}{\ww_i}
\newcommand{\wwj}{\ww_j}
\newcommand{\wwk}{\ww_k}
\newcommand{\ten}[1]{\mathbf{#1}} % boldface for tensors
\newcommand{\BB}{\ten{B}}
\newcommand{\CC}{\ten{C}}
\newcommand{\QQ}{\ten{Q}}
\newcommand{\RR}{\ten{R}}
\renewcommand{\SS}{\ten{S}}
\newcommand{\TT}{\ten{T}}
\newcommand{\VV}{\ten{V}}
\newcommand{\PP}{\mbox{\bf P}}
\newcommand{\WW}{\ten{W}}
\newcommand{\II}{\ten{I}}
\newcommand{\BBij}{\BB_{ij}}
\newcommand{\CCi}{\CC_i}
\newcommand{\QQi}{\QQ_i}
\newcommand{\RRi}{\RR_i}
\newcommand{\SSi}{\SS_i}
\newcommand{\VVj}{\VV_{\!j}} % \! must be *inside* the subscript, not outside
\newcommand{\VVk}{\VV_{\!k}} % \! must be *inside* the subscript, not outside
%\newcommand{\VVdisk}{\VV_\mathrm{\!disk}}
%\newcommand{\VVhalo}{\VV_\mathrm{\!halo}}
\newcommand{\TTij}{\TT_{ij}}
\newcommand{\TTik}{\TT_{ik}}
\newcommand{\T}{^{\scriptscriptstyle \top}}   % transpose
\newcommand{\iT}{^{\scriptscriptstyle -\top}} % inverse transpose
\newcommand{\tr}{\mathrm{tr}}                 % trace
\newcommand{\alphaj}{\alpha_j}
\newcommand{\alphak}{\alpha_k}
%\newcommand{\alphadisk}{\alpha_\mathrm{disk}}
%\newcommand{\alphahalo}{\alpha_\mathrm{halo}}
\newcommand{\qij}{q_{ij}}
\newcommand{\pij}{p_{ij}}
\newcommand{\pik}{p_{ik}}
\newcommand{\qqj}{q_j}
%\newcommand{\subsamplecolor}{ 0.654<(B-V)< 0.685}
\newcommand{\trace}{\mathrm{Trace}}
\newcommand{\norm}{|\!|}
\newcommand{\MLE}{MLE}
\newcommand{\EPhi}{\langle\Phi\rangle}

\begin{document}

\title{
The velocity distribution of nearby stars from \Hipparcos\ data\\
II. Moving groups
}
\author{
Jo~Bovy\altaffilmark{\ref{NYU},\ref{email}}, 
David~W.~Hogg\altaffilmark{\ref{NYU}}, 
and Sam~T.~Roweis\altaffilmark{\ref{Toronto},\ref{Google}}} 
\setcounter{address}{1}
\altaffiltext{\theaddress}{\label{NYU}\stepcounter{address}
Center for Cosmology and Particle Physics, Department of Physics, New York 
University, 4 Washington Place, New York, NY 10003}
\altaffiltext{\theaddress}{\stepcounter{address}\label{email}
To whom correspondence should be addressed: \texttt{jo.bovy@nyu.edu}}
\altaffiltext{\theaddress}{\label{Toronto}\stepcounter{address}
Department of Computer Science, University of Toronto, 
6 King's College Road, Toronto, Ontario, M5S 3G4 Canada}
\altaffiltext{\theaddress}{\label{Google}\stepcounter{address}
?}
\date{\centering{\it Draft, \today}}

\begin{abstract}
We present a three-dimensional reconstruction of the velocity
distribution of nearby stars ($\lesssim 100$ pc) using a maximum
likelihood density estimation technique based on describing the
distribution function as a mixture of Gaussian components. The density
estimation technique is applied to the two-dimensional transverse
velocity measurements from a kinematically unbiased sample of 11,865
main sequence stars observed by the \Hipparcos\ satellite. The
algorithm also accounts for the heterogeneous, covariant noise
properties of the tangential velocity measurements; it returns the
error-deconvolved density. We explore various methods for validating
the complexity of the resulting velocity distribution function,
including how accurately our reconstruction predicts the radial
velocities of a sample of stars from the \gcs\ survey. Thus, we can
quantify the information content of the radial velocity measurements,
which is interesting in the light of the upcoming \Gaia\ mission. We
find that the mean amount of new information gained from a radial
velocity measurement of a single star is small and that most of the
gross outliers are halo stars, the velocity distribution of which is
not well constrained by our sample. We confirm the existence of
``moving groups'' in the velocity distribution of the disk of the
Galaxy, quantifying their statistical significance for the first
time. We find that the color-magnitude diagrams of most of the moving
groups are inconsistent with being trails of evaporating, young
clusters, which favors their interpretation as being due to dynamical
resonances or non-axisymmetry and time-dependence of the disk
potential.  Interestingly, we find traces of various ``thick disk''
and halo streams in this sample of mostly disk stars, although at very
low significance. We also reconstruct the velocity distribution
functions of stars in color-subsamples of our main sample as well as a
sample of giants, and find very similar distributions for all but the
bluest subsample of stars.
\end{abstract}
\keywords{Galaxy: fundamental parameters ---
Galaxy: kinematics and dynamics ---
Galaxy: structure ---
stars: kinematics ---
methods: statistical ---
solar neighborhood
}


\section{Introduction}


Generalities about the velocity distribution [historical, +
theoretical]. Use it to predict radial velocities.

Historical:
\citep{schwarzschild07a,
proctor69a,
1890AN....125...65K,
kapteyn05a,
1906MNRAS..67...34E,
boss08a,
1965gast.conf..111E,
1996AJ....112.1595E,
1996AJ....111.1615E}

Ways to determine the velocity distribution + moments: statistical
deprojection, maximum penalized likelihood, full phase-space
subsamples. Criticize kernel density estimates. Need for proper
probabilistic inference to get to the bottom of this.


Velocity distribution:
\citep{1998A&A...340..384C,
1999A&AS..135....5C,
1998MNRAS.298..387D,
1998AJ....115.2384D,
2005A&A...430..165F,
1999MNRAS.308..731S,
2008A&A...490..135A,
2005ApJ...629..268H}

Methods:
\citep{1990A&A...227..301S,Silverman86a,1996A&AS..117..405L}



%1968MNRAS.139..231W, Woolley and Candy: Theoretical: dispersing star clusters get ripped apart by perturbations
%1970AJ.....75..680I, Innanen and House: Theoretical: oscillations in U and W, stars come together periodically, assumed equal angular momentum for all members

\Hipparcos\ didn't measure radial velocities. \Gaia\ doesn't have many
radial velocities, important to characterize the information content
of radial velocities. In general, data sets at the bleeding edge
always have small signal-to-noise and missing data, need to design
algorithm to deal with this.

\citep{ESA97a}

Biased radial: \citep{1997ESASP.402..473B}

Radial velocity measurements:
\citep{1999A&AS..137..451G,
1999A&AS..135..503G,
1997A&AS..124..255F,
1995A&AS..114..269D,
1995A&AS..110..177D,
2006ARep...50..733B,
2000A&AS..142..217B}


Follow-up radial velocity measurements for Hipparcos were suggested
\citep{1989Msngr..56...12G}.

Train data set on known velocities, or validate, then use the
distribution to infer missing data. 

This paper.

Next paper.







\section{Data, model, and algorithm}


\subsection{Objective function}

\citep{1991PASP..103...95N,1993AJ....105.1455N}

\subsection{Optimization}


\subsection{\Hipparcos\ measurements}
\Hipparcos:
\citep{ESA97a,2007ASSL..250.....V,2007A&A...474..653V}

\Tycho:
\citep{ESA97a,2000A&A...355L..27H,2000A&A...357..367H}





\section{Application to \Hipparcos\ data}


\section{The Geneva-Copenhagen survey}
GCS:
\citep{2004A&A...418..989N,2007A&A...475..519H,2008arXiv0811.3982H}

\section{The complexity of the model}


\section{Information content of the predicted radial velocities}


\section{Other radial velocity data sets}


\bibliographystyle{apj}
\bibliography{EM,bovy_streams}






\end{document}
