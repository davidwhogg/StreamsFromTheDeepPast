
% LaTeX abstract template - please return this template via e-mail to:
% seibel@ari.uni-heidelberg.de
% Deadline: May 01. 2009

\documentclass[]{article}
\usepackage{amssymb,amsmath}
\begin{document}

%-----------------------


%% PLEASE UNCOMMENT one of the following two options:
%%
%% I would like to present a poster.
%%
I would like to give a talk.
%% 
\bigskip

\begin{center} 

{\Large \bf  {Dynamical inference from a kinematic snapshot}}

\end{center}

\begin{center}
{\large \bf   
Jo Bovy, David W.~Hogg, Iain Murray, Hans-Walter Rix, Scott Tremaine
}\\

{\small
CCPP, NYU, New York, NY, United States
}\\
\end{center}

One of the primary objectives of the Gaia mission is to achieve a
coherent understanding of the structure and dynamics of the
Galaxy. However, the problem of inferring the dynamics of the Galaxy
from the Gaia data is fundamentally ill-posed: Gaia will only observe
kinematics---positions and velocities---at a snapshot in time, but it
is only the \emph{accelerations} of the stars that are set by the
gravitational potential. One basis for this non-trivial inference is
the assumption that the potential is (close to) integrable and that (a
large part of) the Galaxy is fully mixed. Under this assumption of
mixed angles the inference becomes a standard exercise in
probabilistic inference, although one that is complicated by large and
heterogeneous measurement uncertainties and missing radial
velocities. We are currently applying this technique to measuring the
mass density distribution in the disk as a function of position (near
the Solar circle)---the so-called ``Oort limit''---which is, in a
sense, a testbed for the larger Gaia problem; we discuss the
``mixed-angle'' approach in this context and present preliminary
results. Despite its usefulness, the mixed-angle assumption is likely
to be incorrect for some components of the Milky Way, because of the
clear abundance of substructure in the halo as well as in the
horizontal motions in the disk---Gaia is expected to uncover much more
substructure. We discuss how the combined information from these
unmixed substructures, e.g., cold stellar streams which trace out
(nearly) a single orbit, in addition to strongly constraining the
potential, could hold clues about its deviations from integrability,
e.g., time-dependence and non-axisymmetry. In the end a combination of
mixed-angle and unmixed substructure techniques into a general
probabilistic method will be necessary to gain a detailed
understanding of the structure and dynamics of the Galaxy from the
Gaia data.


\bigskip
\bigskip

% --------------------------
\end{document}


