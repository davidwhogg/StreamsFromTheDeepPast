
% LaTeX abstract template - please return this template via e-mail to:
% seibel@ari.uni-heidelberg.de
% Deadline: May 01. 2009

\documentclass[]{article}
\usepackage{amssymb,amsmath}
\begin{document}

%-----------------------


%% PLEASE UNCOMMENT one of the following two options:
%%
%% I would like to present a poster.
%%
I would like to give a talk.
%% 
\bigskip

\begin{center} 

{\Large \bf  {The velocity distribution of nearby stars from \emph{Hipparcos} data:
The significance of the moving groups}}

\end{center}

\begin{center}
{\large \bf   
Jo Bovy, David W.~Hogg, Sam T.~Roweis
}\\

{\small
CCPP, NYU, New York, NY, United States
}\\
\end{center}

We present a three-dimensional reconstruction of the velocity
distribution of nearby stars ($\lesssim 100$ pc) using a maximum
likelihood density estimation technique applied to the two-dimensional
tangential velocities of stars. The underlying distribution is modeled
as a mixture of Gaussian components. The algorithm reconstructs the
error-deconvolved distribution function, even when the individual
stars have unique error and missing-data properties. We apply this
technique to the tangential velocity measurements from a kinematically
unbiased sample of 11,865 main sequence stars observed by the
\emph{Hipparcos} satellite.  We explore various methods for validating
the complexity of the resulting velocity distribution function,
including criteria based on Bayesian model selection and how
accurately our reconstruction predicts the radial velocities of a
sample of stars from the Geneva-Copenhagen survey (GCS). Using this
very conservative external validation test based on the GCS, we find
that there is little evidence for structure in the distribution
function beyond the moving groups established prior to the
\emph{Hipparcos} mission. This is in sharp contrast with internal
tests performed here and in previous analyses, which point
consistently to maximal structure in the velocity distribution. We
quantify the information content of the radial velocity measurements
and find that the mean amount of new information gained from a radial
velocity measurement of a single star is significant. This argues for
complementary radial velocity surveys to upcoming astrometric surveys.

\bigskip
\bigskip

% --------------------------
\end{document}


