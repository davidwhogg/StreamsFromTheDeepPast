\documentclass{article}
\usepackage{amssymb,amsmath}

\begin{document}
AJ
\section{Virial Theorem For Logarithmic Potentials}


The Virial is defined as:
\begin{equation}
G := \sum{\mathbf{p_i}\cdot{\mathbf{r_i}}
\end{equation}


By the Virial Theorem, for any time inverval of length $\tau$ in which one's momenta and positions are finite, one obtains the relation:

\begin{equation}\label{vthm}
\frac{1}{\tau}\int_{0}^{\tau}\frac{dG}{dt}dt = \overline{2T}+\overline{\sum{\mathbf{F}_i\cdot\mathbf{r}_i}} = \frac{1}{\tau}[G(\tau)-G(0)]
\end{equation}

provided that the momenta, positions, and potential are finite and $C^1$ over the time period.

For a particle orbiting in a logarithmic potential given by:
\begin{equation}
V(r) = k \log(r)
\end{equation}
We see that
\begin{equation}
\mathbf{F}\cdot\mathbf{r} = -\nabla( k\log(r))\cdot\mathbf{r}=\frac{-k}{r^2}\mathbf{r}\cdot\mathbf{r}= -k.
\end{equation}
 
Thus by \eqref{vthm} we obtain the relation
\begin{equation}
\overline{2T}+\overline{\mathbf{F}\cdot\mathbf{r}} = 2\overline{T} - k = \frac{1}{\tau}[\mathbf{p}(\tau)\cdot\mathbf{r}(\tau)-\mathbf{p}(0)\cdot\mathbf{r}(0)].
\end{equation}
In particular, for closed orbits we obtain the relation:

\begin{equation}
\overline{T}=k/2
\end{equation}

\end{document}